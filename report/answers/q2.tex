\section{Question 2}

The heuristic we designed takes an adaptive greedy approach to find a feasible solution to the problem.
As in the previous question we premcompute the reachability matrix $M_{ij}$ using the Floyd-Warshall algorithm.
The algorithm starts with an empty solution and iteratively assigns a device to a node until a feasible solution is found where all nodes are served by a device.
It keeps track of the nodes that have been searched $N^{Unsearched}$ and the nodes that are not yet served by a device in the solution $N^{Unserved}$.
At each iteration, the algorithm selects the node $i$ that minimizes the ratio
where $c_i$ is the cost of installing a device at node $i$, and the denominator is the number of new nodes that would be served by installing a device at $i$. In other words, we select the node where the cost per newly served node is minimized.
Formally the node to be considered is determined: 
$$\text{select i} \in N^{Unsearched} | min(\frac{c_i}{1 + \sum{k \in N^{Unserved}}^{} M_{ij}})$$

These are the steps taken by the algorithm at each iteration:
\begin{enumerate}
	\item If solution is feasible then stop.
	\item Consider the cheapest node using our defined heuristic.
	\begin{enumerate}[label*=\arabic*.]
		\item If installing a device on that node can serve 1 or more unserved nodes then add it to the solution.
	\end{enumerate}
	\item Go to step 1.
\end{enumerate}

A feasible solution will always be found as in worst case the algorithm will install a device on all nodes and this case is always a feasible solution.

\newpage
\subsubsection*{Results}

\begin{table}[h!]
	\centering
	\begin{tabularx}{\textwidth}{|c|X|c|X|X|}
		\hline
		\textbf{Instance} & \textbf{Solution Cost} & \textbf{o.f. $-$ integer o.f.} & \textbf{Selected nodes} & \textbf{CPU time (s)} \\
		\hline
		1 & 2 & 0 ( 0 \% ) & 2; 5 & 0.000 \\
		\hline
		2 & 36 & 11 ( 44 \%) & 1; 5; 8; 10; 11 & 0.000 \\
		\hline
		3 & 147 & 18 (13.95 \%) &7; 8; 10; 14; 15; 16; 25; 26; 29; 34; 40 & 0.001 \\
		\hline
		4 & 287 & 31 (12.11 \%) & 1; 3; 10; 11; 12; 18; 23; 24; 25; 27; 29; 33; 53; 56; 57; 59; 62; 64; 71; 79; 80 & 0.006 \\
		\hline
	\end{tabularx}
\end{table}

Comparing with the solutions found in the previous section the heuristic solutions are close to the optimal.
Espically for the two big instances 3 and 4 where the heuristic finds a solution that costs $\le 14\%$ of the optimal solution.
And we gain a big improvement in computation time.
The bottleneck of our code really is the precomputation of the reachability matrix which is magnitudes slower than the heuristic.