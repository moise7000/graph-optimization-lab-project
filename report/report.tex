%! Author = ewan
%! Date = 5/26/25

% Preamble
\documentclass[10pt]{article}

% Packages
\usepackage{amsmath}
\usepackage[utf8]{inputenc}
\usepackage{verbatim}
\usepackage{amssymb}
\usepackage{enumitem}
\usepackage{tabularx}
\usepackage{titlesec}
\usepackage{hyperref}

\setcounter{secnumdepth}{0}

\begin{document}

	\title{Graph optimization \\ Lab Project}
	\author{Ewan Decima}
	\date{May 2025}

	\maketitle

	\tableofcontents
	\newpage

	\section{Question 1}

\subsubsection*{Model}
Let's write down the model. First we have the following sets and parameters:
\begin{itemize}
	\item $N$ set of nodes in the network.
	\item $E$ set of edges in the network.
	\item $c_i$ cost of installing a device at node $i \in N$.
	\item $t_{i,j}$ propagation delay on edge $(i,j) \in E$.
	\item $d_{i,j}$ shortest path distance from node $i$ to $j$.
	\item $M = (M_{ij})_{i,j \in N}$ a reachability matrix, such that $M_{ij}$ is equal to one if $d_{i,j} \le T$, zero otherwise.
\end{itemize}

and those decision variables:
\begin{itemize}
	\item $y_i \in \{0,1\}$, a binary variable equal to one if a device is installed at node $i \in N$, zero otherwise.


\end{itemize}

The objective function is then
\begin{equation}
	\min \sum_{i \in N} c_i y_i
\end{equation}

and is subject to the following constraints:
\begin{itemize}
	\item Coverage constraint: Each node must be served by at least one device:
	\[
		\sum_{i \in N } M_{ij}y_i \ge 1 \qquad \forall j \in N
	\]

	
\end{itemize}

\newpage

\subsubsection*{Result}
We obtain the following

\begin{table}[h]
	\centering
	\begin{tabularx}{\textwidth}{|c|c|X|c|}
		\hline
		\textbf{Instance} & \textbf{Optimal value} & \textbf{Selected nodes} & \textbf{CPU time (s)} \\
		\hline
		1 & 2 & 2; 5 & 0.268 \\
		\hline
		2 & 25 & 13; 17; 19 & 0.081 \\
		\hline
		3 & 129 & 1; 7; 13; 15; 21; 22; 28; 29; 34 & 0.127 \\
		\hline
		4 & 256 & 11; 12; 14; 16; 17; 18; 26; 27; 36; 43; 46; 51; 53; 54; 57; 60; 66; 71; 75; 76 & 0.456 \\
		\hline
	\end{tabularx}
	\caption{Results for each instance}
	\label{tab:instance_costs}
\end{table}

	\newpage

	\subsubsection*{Results}

\begin{table}[h!]
	\centering
	\begin{tabular}{|c|c|c|}
		\hline
		\textbf{Instance} & \textbf{Optimal value} & \textbf{Solver CPU time (s)} \\
		\hline
		1 & 10  & 0.000 \\
		\hline
		2 & 47 & 0.000 \\
		\hline
		3 & 137 & 0.000 \\
		\hline
		4 & 285 & 0.000 \\
		\hline
	\end{tabular}
	\label{tab:instance_costs}
\end{table}
	\newpage

	\section{Question 3}

This problem can be represented as a \textit{set covering problem}.  \\

To convert our problem into a \textit{set covering problem} we precompute the shortest distance between each pair of nodes in the graph, which can be easily done using the Floyd-Warshall algorithm in $\mathcal{O}(n^3)$ time, where $n$ is the number of nodes in the graph.
Then we construct the reachability matrix by checking the distance between each pair of nodes.

\begin{align*}
	M \in \mathcal{M}_{n \times n} (\{0, 1\})  \\
	M_{ij} = \begin{cases}
		1 & \text{if } d_{ij} \leq T \\
		0 & \text{otherwise}
	\end{cases}
\end{align*}

We can solve the \textit{set covering problem} using the rows of the reachability matrix as the possible subsets to obtain the solution to the original problem.

	\newpage

	\subsection*{Question 4}

\subsubsection*{Parameters}

\begin{itemize}
	\item $G = (N, A)$: the directed graph with nodes $N$ and arcs $A$.
	\item $d_k$: data to send from node $k \in N$.
	\item $c_i$: cost of installing devices at node $i \in N$.
	\item $cap_i$: capacity of devices installed at node $i \in N$.
	\item $T$: maximum allowed propagation time for data.
	\item $t_{ij}$: propagation delay on arc $(i, j) \in A$.
	\item $g_{ij}$: cost of installing channels on arc $(i, j) \in A$.
	\item $u$: capacity of each channel installed on arcs $(i, j) \in A$.
\end{itemize}

\subsubsection*{Decision Variables}

\begin{itemize}
	\item $x_{ij}$ binary: one if data is sent from node $i$ to node $j$, zero otherwise.
	\item $y_{ij}^k$ binary: one if data from node $k$ is sent on arc $(i, j)$, zero otherwise.
	\item $z_i$ integer: number of devices installed at node $i$.
	\item $w_{ij}$ integer: number of channels installed on arc $(i, j)$.
\end{itemize}

\subsubsection*{Constraints}

\begin{enumerate}
	\item Each node must send its data to exactly one other node.
	\item For each node, enough devices must be installed to handle the data sent to it.
	\item For each arc, enough channels must be installed to handle the data sent on that arc.
	\item The propagation time must not exceed the maximum allowed time.
	\item Flow constraints: ensure that one and only one path from node $k$ to node $l$ exists if $x_{kl} = 1$; ensure that no data is sent if a node sends data to itself.
\end{enumerate}

\subsubsection*{Model}

\begin{align*}
    \min \quad & \sum_{i \in N} c_i z_i + \sum_{(i, j) \in A} g_{ij} w_{ij} & \\
	\\
    \text{s.t.} \quad &  \\
	1. \quad & \sum_{j \in N} x_{ij} = 1 \quad \forall i \in N  \\
	2. \quad & \sum_{i \in N} d_i x_{ij} \le cap_j \; z_j \quad \forall j \in N  \\
	3. \quad & \sum_{k \in N} d_k y_{ij}^k \le u \; w_{ij} \quad \forall (i, j) \in A \quad  \\
	4. \quad & \sum_{(i, j) \in A} y_{ij}^k t_{ij} \le T \quad \forall k \in N  \\
	5. \quad & \sum_{j \in N \mid (i, j) \in A} y_{ij}^s \; - \sum_{j \in N \mid (j, i) \in A} y_{ji}^s = x_{st} \cdot \left\{
	\begin{array}{rl}
		0  & \text{if } s = t \vee i \notin\{s, t\}  \\
		1  & \text{if } s \ne t \land i = s  \\
		-1 & \text{if } s \ne t \land i = t
	\end{array}
	\right.
	\quad \forall i \in N, \; s \in N, \; t \in N  \\
	\\
	& x_{ij} \in \{0, 1\}  \\
	& y_{ij}^k \in \{0, 1\}  \\
	& z_i \in \mathbb{Z}^+  \\
	& w_{ij} \in \mathbb{Z}^+
\end{align*}

	\newpage


    \subsection*{Question 5}
    \textit{How can the cutset inequalities for network design be applied to the problem described in 4? Generate all the single node cutset-based inequalities (those that consider the cuts separating each single node from the others). Add them to the formulation and compute the continuous relaxation with and without them. Compare the two relaxations solving the instances. Are the inequalities effective? Upload the \texttt{.run} and \texttt{.mod} files, and a \texttt{.pdf} file with the updated model and the comparison of the two continuous relaxations.}

    \section*{Question 6}
    \textit{How can the cover inequalities for the knapsack problem be applied to the problem described in 4? Generate heuristically some cover-based inequalities, add them to the formulation and compute the continuous relaxation with and without them for all the instances. Are they effective? Upload the \texttt{.run} and \texttt{.mod} files, and a \texttt{.pdf} with the results and the updated model.}

\end{document}




